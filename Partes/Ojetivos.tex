\section{Objetivos}
\label{objetivos}

%Assumindo que existe um problema a ser resolvido, apresente qual o objetivo de seu projeto de pesquisa. O que você pretende (ou pretendeu) exatamente fazer. Aqui, deve aparecer a principal ``contribuição'' de seu projeto. Qual é a principal ``coisa'' que você pretende/pretendeu fazer? Qual sua principal entrega? Não é necessário criar uma subseção para cada tipo. Pode haver uma única seção, chamada de ``objetivos'' cujo texto divida-se naturalmente em objetivo geral e objetivos específicos, deixando claro qual caso está sendo tratado em cada momento. Para diferenciar o objetivo geral dos objetivos específicos, siga as seguintes diretrizes:

\begin{itemize}
		\item \textbf{Objetivo geral}: Desenvolver um jogo educativo que auxilie o ensino de matemática matemática. 
		
		\item \textbf{Objetivos específicos}: Analisar se jogos eletrônicos devem ser usados em pró da educação infantil, mostrar exemplos de jogos educativos passados.
\end{itemize}
