\chapter{\uppercase{Introdução}}
\label{introducao}


 Segundo \citeonline{einstein}, é dito que nem todos tem a mesma velocidade de aprendizado, alguns aprendem com mais facilidade, e outros não. Essa dificuldade em aprender é complexa. Ela pode ter fatores comportamentais ou neurológicos, pode ter várias origens dentro ou fora da escola, podendo surgir em qualquer ponto da vida, e pode ser maior ainda se o aluno possuir transtornos como deficit de atenção, hiperatividade, dislexia ou falta de auto-estima. Deficiências físicas como problemas de visão e audição também podem atrapalhar, e algo simples como nascer com um cérebro que processa informações mais lentamente já pode impactar na aprendizagem de qualquer assunto e prática. O professor deve sempre levar em conta essa dificuldade, principalmente durante o ensino fundamental, onde as crianças estão aprendendo habilidades fundamentais para a vida adulta, como por exemplo, a matemática, uma disciplina fundamental, mas que porém, possui altos níveis de reprovação.

Assim como \citeonline{einstein} disse, as crianças aprendem e memorizam melhor brincando, ou se envolvendo em atividades estimulantes, como brincadeiras, onde podem colocar o seu conhecimento em prática enquanto se divertem, podendo errar e tentar novamente quantas vezes quiserem. As brincadeiras desempenham funções psicossociais, afetivas e intelectuais básicas no processo de desenvolvimento infantil, criando um ambiente propicio á aprendizagem que satisfaz a necessidade de brincar da criança e estimula a curiosidade e interações sociais de forma prazerosa.Uma outra forma existente de educar desta maneira são videogames, ou jogos eletrônicos. Com a popularidade de computadores e dispositivos eletrônicos de qualquer tipo no dia a dia, eles estão se tornando cada vez mais comuns na sala de aula, inclusive jogos eletrônicos, que são usados em sala de aula como uma ferramenta auxiliar de ensino já faz algum tempo, implementados de diversas formas para tornar a aprendizagem mais fácil e interessante pra os alunos, sendo especialmente úteis para os alunos que possuem dificuldade de aprender uma disciplina tão importante como matemática.

Neste Trabalho de Conclusão de Curso, o objetivo principal é mostrar, por meio de uma pesquisa descritiva, se usar jogos eletrônicos como uma ferramenta de ensino complementar no ensino de  matemática para crianças do ensino fundamental é justificável e como objetivo secundário criar um jogo 2D educativo sobre matemática que consiga ensinar sobre adição, subtração, a diferença entre números pares e ímpares e a existência de números negativos para crianças do ensino fundamental enquanto elas jogam um produto memorável e divertido.