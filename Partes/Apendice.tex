\chapter{\uppercase{Considerações finais}}
\label{conclusao}

%Dicas para desenvolver a estrutura das considerações finais:

%\begin{itemize}

%\item Contextualize o projeto desenvolvido, ou seja, ajude o leitor a "relembrar" a estrutura. 

%\item Apresente os resultados.

%\item Apresente as contribuições do projeto para o meio acadêmico e/ou para o mercado. 

%\item Se for possível, descreva sobre melhorias para próximas pesquisas.

%\end{itemize}

%Neste Capítulo pode ser inserido as Seções: Trabalhos Futuros, Contribuições do Projeto, Artigos publicados. 

 Antes de iniciar este Trabalho de Conclusão de Curso, já havia diversos relatos falados vindos de conversas e palestras anteriores de como jogos eletrônicos eram importantes para outras atividades além de divertir, inclusive a pedagogia, na qual já se sabia que este conceito era experimentado faz alguns anos por conta da existência de jogos educativos. Ao desenvolver a pesquisa feita aqui, o objetivo inicial era confirmar estes relatos, usando a área de educação como exemplo, além de desenvolver o jogo educativo em paralelo mencionado anteriormente para treinar o desenvolvimento futuro de jogos e fazer um jogo educativo de matemática que pode possivelmente ser distribuído gratuitamente em escolas públicas.
 
 O desenvolvimento do jogo educativo ``Number Invasion" demorou, mas foi uma experiência enriquecedora. Foi um prazer trabalhar nele por conta da pretensão de seguir a área de desenvolvimento de jogos eletrônicos no futuro e adquirir experiência em um projeto como este era fundamental. Mais conhecimento sobre o Unity e a linguagem de programação C\# foi adquirido conforme o projeto foi se realizando, e diversas dúvidas sobre como alguns algoritmos eram feitos foram respondidas graças a enorme comunidade do Unity localizadas em fóruns e vídeos pela Internet, e ``Number Invasion" então se tornou o primeiro jogo completo feito, o que deu experiência valiosa e abre as portas para projetos futuros sem cunho educacional.
 
 A pesquisa foi a parte mais difícil deste Trabalho de Conclusão de Curso, pois explorava um campo novo onde não se havia experiência prévia como foi o desenvolvimento do jogo. Conforme a pesquisa se desenvolveu, o assunto foi aprendido aos poucos e o material da pesquisa foi complementando o jogo educativo e, com os dois unidos ao final da pesquisa, um completava o outro, onde a pesquisa mostrava a parte teórica e o jogo a parte prática.
 
 Com tudo o que foi apresentado ao decorrer deste Trabalho de Conclusão de Curso, conclui-se que os objetivos foram alcançados. O jogo educativo feito em paralelo, apesar de ter destino incerto quanto a ser distribuído em escolas, foi finalizado com sucesso sem apresentar erros e foi um bom treino para projetos futuros, e todas as evidências apresentadas de como o cérebro é afetado por jogos eletrônicos mostram que os mesmos têm grande potencial de serem usados como ferramentas auxiliares em diversas áreas, inclusive a pedagogia. Porém, é necessário estudar antes como será a implementação dos jogos eletrônicos em sala de aula, para que os mesmos sejam eficientes em sua função de ensinar e o máximo de conhecimento seja extraído.
 
  \section{Trabalhos Futuros}
  
  Este jogo não foi testado em sala de aula, portanto, sua capacidade de ensinar e ser divertido é desconhecida. Serão necessários testes práticos no futuro para comprovar sua eficácia.